\documentclass[11pt]{article}

\title{matrix identities}
\author{sam roweis}
\date{(revised June 1999)}

\usepackage{amsmath}
\usepackage{amssymb}
\usepackage{amsthm}

%\usepackage[top=1.5in, bottom=1.5in, nohead]{geometry}
\usepackage[nohead]{geometry}

% Change vectors to bolded text rather than the arrow notation.
\renewcommand{\vec}[1]{\mathbf{#1}}

% Define a new command for matrices to do the same thing.
\newcommand{\mat}[1]{\mathbf{#1}}

% Reappropriate \t for a capital T
\renewcommand{\t}{T}

% Define \inv for the '-1' superscript
\newcommand{\inv}{{-1}}

% Set things up to match Sam's settings for equation numbering. 
\numberwithin{equation}{subsection}
\renewcommand{\theequation}{\arabic{subsection}\alph{equation}}

\parindent 0in

\begin{document}
\maketitle

\textbf{note} that $\vec{a}$, $\vec{b}$, $\vec{c}$ and $\mat{A}$, 
$\mat{B}$, $\mat{C}$ do not depend on $\mat{X}$, $\mat{Y}$, 
$\vec{x}$, $\vec{y}$, or $z$.

\subsection{basic formulae}
\begin{align}
\mat{A}(\mat{B} + \mat{C}) & = \mat{AB} + \mat{BC} & \\
(\mat{A} + \mat{B})^\t & = \mat{A}^\t + \mat{B}^\t &\\
(\mat{AB})^\t & = \mat{B}^\t\mat{A}^\t & \\
(\mat{AB})^{-1} & = 
	\mat{B}^\inv \mat{A}^\inv \ \text{(if individual inverses exist)} \\
(\mat{A}^\inv)^\t & = (\mat{A}^\t)^\inv
\end{align}
\subsection{trace, determinant and rank}
\begin{align}
|\mat{AB}| & = |\mat{A}||\mat{B}|  \\
|\mat{A}^\inv| & = \frac{1}{|\mat{A}|}  \\
|\mat{A}| & = \prod \text{eigenvalues}  \\
\mathrm{Tr}(\mat{A}) & = \sum \text{eigenvalues}  \\
\text{if the cyclic products are well-defined, } & \notag \\
\mathrm{Tr}[\mat{A}\mat{B}\mat{C} \ldots] = \mathrm{Tr}[\mat{B}\mat{C} 
\ldots \mat{A}] & = \mathrm{Tr}[\mat{C} \ldots \mat{A}\mat{B}] = \ldots   \\
\mathrm{rank}[\mat{A}] = \mathrm{rank}[\mat{A}^\t ] 
& = \mathrm{rank}[\mat{A}^\t \mat{A}] = \mathrm{rank}[\mat{AA}^\t]  \\
\text{condition number} & = \gamma = 
	\sqrt{\frac{\text{biggest eigenvalue}}{\text{smallest eigenvalue}}} 
\end{align}

\textbf{derivatives} of scalar forms with respect to scalars,
vectors, or matrices are indexed in the obvious way.
similarly, the indexing for derivatives of vectors and matrices
with respect to scalars is straightforward.
\subsection{derivatives of traces}
\begin{align}
% d(Tr[X]) / dX = I
\frac{
	\partial \mathrm{Tr}[\mat{X}]
}{
	\partial \mat{X}
} & = \mat{I} \\
%  d Tr[XA]/dX = dTr[AX]/dX = A^T
\frac{
	\partial \mathrm{Tr}[\mat{XA}]
}{
	\partial \mat{X}
} = 
\frac{
	\partial \mathrm{Tr}[\mat{AX}]
}{
	\partial \mat{X}
} & = \mat{A}^\t \\
% dTr(X^tA)/dX = dTr(AX^T)/dX = A
\frac{
	\partial \mathrm{Tr}[\mat{X}^\t\mat{A}]
}{
	\partial \mat{X}
} = 
\frac{
	\partial \mathrm{Tr}[\mat{AX}^\t]
}{
	\partial \mat{X}
} & = \mat{A} \\
% dTr(X^T A X) / dX
\frac{
	\partial \mathrm{Tr}[\mat{X}^\t \mat{A} \mat{X} ]
}{
	\partial \mat{X}
} & = (\mat{A} + \mat{A}^\t)\mat{X} \\
% dTr(X^-1 A) / dX = -X^-1 A X^-1
\frac{
	\partial \mathrm{Tr}[\mat{X}^\inv \mat{A}]
}{
	\partial \mat{X}
} & = -\mat{X}^\inv \mat{A}^\t \mat{X}^\inv
\end{align}
\subsection{derivatives of determinants}
\begin{align}
% d AXB/dX = |AXB|(X^-1)^T = |AXB|(X^T)^-1
\frac{
	\partial |\mat{AXB}|
}{
	\partial \mat{X}
} & =
|\mat{AXB}|(\mat{X}^\inv)^\t = 
|\mat{AXB}|(\mat{X}^\t)^\inv \\
% d ln|X| / dX = (X^-1)^T = (X^T)^-1
\frac{
	\partial \ln |\mat{X}|
}{
	\partial \mat{X}
} & =
(\mat{X}^\inv)^\t = (\mat{X}^\t)^\inv \\
%
\frac{
	\partial \ln \mat{X}(z)
}{
	\partial z
} & = \mathrm{Tr}
\left[
		\mat{X}^\inv \frac{\partial \mat{X}}{\partial z}
\right] \\
%
\frac{
	\partial |\mat{X}^\t \mat{A} \mat{X}|
	}{
	\partial \mat{X}
	} & = 
	|\mat{X}^\t \mat{AX}| 
	\left(
		\mat{AX} (\mat{X}^\t \mat{AX})^\inv +
	\mat{A}^\t \mat{X} (\mat{X}^\t \mat{A}^\t \mat{X}
	\right)^\inv)
\end{align}
\subsection{derivatives of scalar forms}
\begin{align}
%
\frac{
	\partial \left( \vec{a}^\t \vec{x} \right)
}{
	\partial \vec{x}
} & = \frac{
	\partial \left( \vec{x}^\t \vec{a} \right)
}{
	\partial \vec{x}
} = \vec{a} \\
%
\frac{
	\partial \left(\vec{x}^\t \mat{A} \vec{x} \right)
}{
	\partial \vec{x}
} & = \left(\mat{A} + \mat{A}^\t\right) \vec{x} \\
%
\frac{
	\partial \left(\vec{a}^\t \mat{X} \vec{b}\right)
}{
	\partial \mat{X}
} & = \vec{a}\vec{b}^\t \\
%
\frac{
	\partial \left(\vec{a}^\t \mat{X}^\t \vec{b}\right)
}{
	\partial \mat{X}
} & = \vec{b}\vec{a}^\t \\
% 
\frac{
	\partial \left(\vec{a}^\t \mat{X} \vec{a}\right)
}{
	\partial \vec{X}
} & =
\frac{\partial \left(
		\vec{a}^\t \mat{X}^\t \vec{a}
	\right)
}{
	\partial \mat{X}
} =
\vec{a}\vec{a}^\t \\
%
\frac{
	\partial \left(\vec{a}^\t \mat{X}^\t \mat{C} \mat{X} \vec{b}\right)
}{
	\partial \mat{X}
} & = \mat{C}^\t \mat{X} \vec{a}\vec{b}^\t + 
	\mat{C} \mat{X} \vec{b}\vec{a}^\t \\
%
\frac{\partial \left(
		\left( 
			\mat{X} \vec{a} + \vec{b}
		\right)^\t 
		\mat{C}
		\left(
			\mat{X} \vec{a} + \vec{b}
		\right)
	\right)
	}{
	\partial \mat{X}
	} & = 
	\left(
		\mat{C} + \mat{C}^\t
	\right)
	\left(
		\mat{X}\vec{a} + \vec{b}
	\right)\vec{a}^\t
\end{align}


the \textbf{derivative} of one vector $\vec{y}$ with respect to another vector
$\vec{x}$ is a matrix whose $(i, j)^{th}$ element is $\partial y_j / 
\partial x_i$. such a derivative should be written as $\partial \vec{y}^\t / 
\partial \vec{x}$ in which case it is the \textit{Jacobian matrix} of 
$\vec{y}$ wrt $\vec{x}$. its determinant represents the ratio of the 
hypervolume $d\vec{y}$ to that of $d\vec{x}$ so that $\int f(\vec{y}) 
d\vec{y} = \int f(\vec{y}(\vec{x}))|\partial \vec{y}^\t / \partial \vec{x}|
d \vec{x}$. however, the sloppy forms $\partial \vec{y} / \partial \vec{x}$, 
$\partial \vec{y}^\t / \partial \vec{x}^\t$ and $\partial \vec{y} / 
\partial \vec{x}^\t$ are often used for this Jacobian matrix.

\subsection{derivatives of vector/matrix forms}

\begin{align}
\frac{
	\partial \left(\mat{X}^\inv\right)
}{
	\partial z
} & = -\mat{X}^\inv 
\frac{
	\partial \mat{X}
}{
	\partial z
} \mat{X}^\inv \\
%
\frac{
	\partial \left( \mat{A} \vec{x} \right)
}{
	\partial z
} & = \mat{A} 
\frac{
	\partial \vec{x}
}{
	\partial z
} \\
%
\frac{
	\partial \left(\mat{XY}\right)
}{
	\partial z
} & = \mat{X} \frac{\partial \mat{Y}}{\partial z} + 
	\frac{\partial \mat{X}}{\partial z}\mat{Y} \\
%
\frac{
	\partial \left(\mat{AXB}\right)
}{
	\partial z
} & = \mat{A} \frac{\partial \mat{X}}{\partial z} \mat{B} \\
%
\frac{
	\partial \left( \vec{x}^\t \mat{A} \right)
}{
	\partial \vec{x}
} & = \mat{A} \\
%
\frac{
	\partial \left( \vec{x}^\t \right)
}{
	\partial \vec{x}
} & = \mat{I} \\
%
\frac{
	\partial \left(\vec{x}^\t \mat{A} \vec{x} \vec{x}^\t \right)
}{
	\partial \vec{x}
} 
= & \left(\mat{A} + \mat{A}^\t\right)\vec{x}\vec{x}^\t + 
\vec{x}^\t \mat{A} \vec{x} \mat{I}
\end{align}
\subsection{constrained maximization}
the maximum over $\vec{x}$ of the quadratic form
\begin{align}
	\vec{\mu}^\t \vec{x} - \frac{1}{2} \vec{x}^\t \mat{A}^\inv \vec{x}
\end{align}
subject to the $J$ conditions $c_j(\vec{x}) = 0$ is given by:
\begin{align}
\mat{A}\vec{\mu} + \mat{A}\mat{C}\mat{\Lambda}, & & \ \ 
\mat{\Lambda} = -4 \left(\mat{C}^\t \mat{A} \mat{C}\right) 
\mat{C}^\t \mat{A} \vec{\mu}
\end{align}
where the $j^{th}$ column of $\mat{C}$ is $\partial c_j (\vec{x}) /
\partial \vec{x}$
\subsection{symmetric matrices}
have real eigenvalues, though perhaps not distinct and can always be
diagonalized to the form 
\begin{align}
	\mat{A} & = \mat{C}\mat{\Lambda}\mat{C}^\t \tag{8}
\end{align}

where the columns of $\vec{C}$ are (orthonormal) eigenvectors (i.e.
$\mat{CC}^\t = \mat{I}$) and the diagonal of $\mat{\Lambda}$ has the
eigenvalues

\subsection{block matrices}

for conformably partitioned block matrices, addition and multiplication is
performed by adding and multiplying blocks in exactly the same way as scalar
elements of regular matrices.

however, determinants and inverses of block matrices are very tricky: for 
2 blocks by 2 blocks the results are:

\begin{align}
	\left|
	\begin{array}{cc}
		\mat{A}_{11} & \mat{A}_{12} \\
		\mat{A}_{21} & \mat{A}_{22} 
	\end{array} \right| & = |\mat{A}_{22}| \cdot |\mat{F}_{11}|
	= |\mat{A}_{11}| \cdot |\mat{F}_{22}| \\
	\left[ \begin{array}{cc}
		\mat{A}_{11} & \mat{A}_{12} \\
		\mat{A}_{21} & \mat{A}_{22} 
	\end{array} 
	\right]^\inv & =
	\left[
	\begin{array}{cc}
	\mat{F}_{11}^\inv & -\mat{A}_{11}^\inv \mat{A}_{12} \mat{F}_{22}^\inv \\
	- \mat{F}_{22}^\inv \mat{A}_{21} \mat{A}_{11}^\inv & \mat{F}_{22}^\inv 
	\end{array}
	\right] \\
	& = 
	\left[ 
	\begin{array}{cc}
	% (1, 1)
	\mat{A}_{11}^\inv + \mat{A}_{11}^\inv \mat{A}_{12}
	\mat{F}_{22}^\inv \mat{A}_{21} \mat{A}_{11}^\inv 
	&
	% (1, 2)
	- \mat{F}_{11}^\inv \mat{A}_{12} \mat{A}_{22}^\inv
	\\
	% (2, 1)
	- \mat{A}_{22}^\inv \mat{A}_{21} \mat{F}_{11}^\inv
	&
	% (2, 2)
	\mat{A}_{22}^\inv + \mat{A}_{22}^\inv \mat{A}_{21} \mat{F}_{11}^\inv
	\mat{A}_{12} \mat{A}_{22}^\inv
	\end{array}
	\right] \notag
\end{align}
where 
\begin{align*}
\mat{F}_{11} = \mat{A}_{11} - \mat{A}_{12} \mat{A}_{22}^{-1} \mat{A}_{21}
& &
\mat{F}_{22} = \mat{A}_{22} - \mat{A}_{21} \mat{A}_{11}^{-1} \mat{A}_{12}
\end{align*}

for \textit{block diagonal} matrices things are much easier:
\begin{align}
\left| 
	\begin{array}{cc}
		\mat{A}_{11}  & \mat{0}     \\
		\mat{0}       & \mat{A}_{22}
	\end{array}
\right| & = |\mat{A}_{11}||\mat{A}_{22}| \\
\left[
	\begin{array}{cc}
		\mat{A}_{11}  & \mat{0}      \\
		\mat{0}       & \mat{A}_{22}
	\end{array}
\right] & = 
\left[
	\begin{array}{cc}
		\mat{A}_{11}^\inv & \mat{0} \\
		\mat{0}           & \mat{A}_{22}^\inv
	\end{array}
\right]
\end{align}

\subsection{matrix inversion lemma (sherman-morrison-woodbury)}
using the above results for block matrices we can make some
substitutions and get the following important results:
\begin{align}
(\mat{A} + \mat{XBX}^\t)^\inv & = \mat{A}^\inv - \mat{A}^\inv \mat{X}
	(\mat{B}^\inv + \mat{X}^\t \mat{A}^\inv \mat{X})^\inv \mat{X}^\t 
	\mat{A}^\inv \\
|\mat{A} + \mat{X}\mat{B}\mat{X}|^\t & = |\mat{B}||\mat{A}|
	|\mat{B}^\inv + \mat{X}^\t \mat{A}^\inv \mat{X}|
\end{align}
where $\mat{A}$ and $\mat{B}$ are \textit{square} and \textit{invertible}
matrices but need not be of the same dimension. this lemma often allows 
a really hard inverse to be converted into an easy inverse. the most 
typical example of this is when $\mat{A}$ is large but diagonal, and
$\mat{X}$ has many rows but few columns.
\end{document}
